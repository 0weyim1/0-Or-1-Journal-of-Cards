\documentclass[a4paper,12pt]{article}
\usepackage[margin=1in,left=1in,includefoot]{geometry}
\usepackage[margin=1in,left=1in,includefoot]{geometry}
\usepackage[tbtags]{amsmath}%%%%%
%%%%%%%%%%%%%%%%%%%%%%%%%
 \usepackage{amsmath}
 \usepackage{amsfonts}
 \usepackage{amssymb}
\usepackage{mathtools}
\usepackage{amsthm}
\everymath{\displaystyle}
\usepackage[T1]{fontenc}
\usepackage{mathpazo}%%palatino font for text
\usepackage[euler-digits,euler-hat-accent]{eulervm}%euler for math font
%%%%%%%%%%%%%%%%%%%%%%%%%%%%%%%%%%%%
%%%%%%%%%%%%%%%%%%%%%%%%%%%%%%%%%%%%
\usepackage{ragged2e}%%%%%%%%%%%%%%for \justify
%%%%%%%%%%%%%%%%%%%%%%%%%%%%
\usepackage[dvipsnames]{xcolor}
\usepackage{xcolor}
\usepackage{booktabs,tabularx}
\usepackage{multirow}
\usepackage{tikz}
\usepackage{graphicx} % Required for including images
\usepackage[font=small,labelfont=bf]{caption} % Required for specifying captions to tables and figures
%%%%%%%%%%%%%%%%%%%%%%%%%%%%%%%%
\usepackage[colorlinks=true]{hyperref}%
 %%%%%%%%%%%%%%%%%%%%%%%%%%%%%%%%%%%%%
\usepackage{fontspec}
\usepackage[ethiop,main=english]{babel}
\newfontface{\geezfont}{FreeSerif}
\newenvironment{geez}{\geezfont}{}
\lccode`፡=`፡  \catcode`፡=11
\lccode`።=`። \catcode`።=11
\babelprovide[import,
  onchar = fonts ids,
  typography/intraspace = 0 .1 0,
  typography/linebreaking = s, 
  characters/ranges = 1200..139F 2D80..2DDF AB00..AB2F,
  ]{amharic}
\babelfont[amharic]{rm}{FreeSerif}
%%%%%%%%%%%%%%%%%%%%%%%%%%%%%%%%%%%%%%%%%%%%%%%%
%%%%%%%%%%%%%%%%%%%%%%%%%%%%%%%%%%%%%%%%%%%%%%%%%%%%
\newtheoremstyle{mystyle}%                % Name
  {}%                                     % Space above
  {}%                                     % Space below
  {\itshape}%                             % Body font
  {}%                                     % Indent amount
  {\bfseries}%                            % Theorem head font
  {.}%                                    % Punctuation after theorem head
  { }%                                    % Space after theorem head, ' ', or \newline
  {}%                                     % Theorem head spec (can be left empty, meaning `normal')
%%%%%%%%%%%%%%%%%%%%%%%%%%%%%%%%%%%%%%%%%%%%%%%%%%%%%%%%%%%%%%%%%%%%%
\theoremstyle{mystyle}
\newtheorem{theorem}{Theorem}
\newtheorem{proposition}{Proposition}
\newtheorem{lemma}{Lemma}
\newtheorem{corollary}{Corollary}
\newtheorem{example}{Example}
\newtheorem{solution}{Solution}
\newtheorem{conclusion}{Conclusion}
\newtheorem{definition}{Definition}
\newtheorem{remark}{Remark}
\newtheorem{amharicdefinition}{\begin{geez}ትርጉም\end{geez}}
%%%%%%%%%%%%%%%%%%%%%%%%%%%%%%%
%%%%%%%%%%%%%%%%%%%%%%%%%%%%%%%%%%%%%%%%%%%%%%%%%
\numberwithin{equation}{section}
\numberwithin{theorem}{section}
\numberwithin{proposition}{section}
\numberwithin{example}{section}
\numberwithin{remark}{section}
\numberwithin{lemma}{section}
\numberwithin{corollary}{section}
\numberwithin{definition}{section}
\numberwithin{amharicdefinition}{section}
%%%%%%%%%%%%%%%%%%%%%%%%%%%%%%%%%%%%%%%%%%%%%%%%%%
%%%%%%%%%%%%%%%%%%%%%%%%%%
\usepackage[shortlabels]{enumitem}
\usepackage{soul}%%for highlighting texts and equations begin inside $$..\hl{some text here}
\newcommand{\mathcolorbox}[2]{\colorbox{#1}{$\displaystyle #2$}}%%to highlight math equations which is inside \begin{equation}...\end{equation}
\usepackage{authblk}
\title{
{\large General Knowledge 0.4 For Pin Number 6}
}
\author[1,2,$*$]{\small Dagnachew Jenber}
%\author[2]{second author}
%\author[3]{third author}
%\author[4]{fourth author}
\affil[1]{ Department of Mathematics, Bahir Dar University, Bahir Dar, Ethiopia.}
\affil[2]{Department of Mathematics, Addis Ababa Science and Technology University, Addis Ababa, Ethiopia.}
%\affil[3]{third affilation}
%\affil[4]{fourth affilation}
\affil[$*$]{Corresponding author: Dagnachew Jenber, dagnachew.Jenber@aastu.edu.et}
%\date{}                     %% if you don't need date to appear
\setcounter{Maxaffil}{0}
\renewcommand\Affilfont{\itshape\small}
\usepackage{xhfill}
\setlength{\parindent}{0pt}
%%%%%%%%%%%%%%%%%%%%%%%%%%%%%%%%%%%%%%%%%%%%%
%%%%%%%%%%%%%%%%%%%%%%%%%%%%%%%%%%%%%%%%%%%%%%%%%%%%%%%%%%
\usepackage[backend=bibtex,maxnames=1000,minnames=10,maxalphanames=1000,
minalphanames=10,style=numeric,sorting=anyt,firstinits=true]{biblatex}
\DeclareNameAlias{default}{last-first}
\addbibresource{gk0.4p6.bib}
\renewbibmacro{in:}{}
%%%%%%%%%%%%%%%%%%%%%%%%%%%%%
%%%%%%%%%%%%%%%%%%%%%%%%%
\usepackage{geometry}
\usepackage{graphicx}
\makeatletter         
\def\@maketitle{
\raggedright
\begin{center}
{\large \bfseries \sffamily \@title }\\[1.5ex]
{  \@author}\\[8ex]
\end{center}}
\makeatother
%%%%%%%%%%%%%%%%%%%%%%%%%%%%%%%%%%%%%%%%%%%%%%%
\makeatletter
\renewcommand\tableofcontents{%
  \null\hfill\textbf{\Large\contentsname}\hfill\null\par
  \@mkboth{\MakeUppercase\contentsname}{\MakeUppercase\contentsname}%
  \@starttoc{toc}%
}
\makeatother
%%%%%%%%%%%%%%%%%%%%%%%%%%%%%%%%%%%%%%%%%%%%%%%%%%%
%%%%%%%%%%%%%%%%%%%%%%%%%%%%%%%%%%%%%%%%%%%%%%%%%%%%%
\usepackage{hyperref}% http://ctan.org/pkg/hyperref
%%%%%%%%%%%%%%%%%%%%%%%%%%%%%%%%%%%%

%%%%%%%%%%%%%%%%%%%%%%%%%%%%%%%%%%%%%%
%%%%%%%%%%%%%%%%%%%%%%%%%%%%%%%%%%%%%%%%
\begin{document}
\maketitle
\fontfamily{kpfonts}
\hypersetup{
  colorlinks,
  citecolor=red,
  linkcolor=red,
  urlcolor=blue}

  \hypersetup{
  citebordercolor=red,
  filebordercolor=red,
  linkbordercolor=blue
}
\centering
{\bf Abstract}
\justify
This work presents 38 number of cards from different discplines focused on english, physics and mathematics subject. The jester cards are extraneous, Provocative, Brusque, Momentum, Fraction Addition Formula and Equation of a circle.
\section{\begin{geez}መግቢያ\end{geez}}
\label{S:2}
አሁን ባለንበት ዘመን የአንባቢያን ማህበረሰብ እየቀነሰ መምጣት አሳሳቢ ደረጃ ላይ ደርሷል። በብዙ ምክኒያት ሰወች ቁጭ ብለው
ማንበብ የተውበት ጊዜ ነው። ለምሳሌ ጠቃሚ ያልሆነ ሶሻል
ሚዲያ ላይና በአልባሌ ቦታወች ጊዜን ማጥፋት ከብዙወቹ ትንሾቹ ምክኒያቶች ናቸው። በ2017 ዓ.ም ዳኛቸው ለዚህ የሚሆን መፍትሄ ብሎ ያቀረበው 0 ወይም 1 ጨዋታ በሚል ርእስ የተዘጋጀ ትልቅ አክሲዮን ማህበር አለ። ይህ አክሲዮን ማህበር ከላይ የተጠቀሰውን ችግር በሚከተሉት መልኩ መፍታት ይቻላል ብሎ ያምናል። በዚህ ፅሁፍ ውስጥ የተካተተው መፍትሄ አሳማኝ ሆኖ አግኝተነዋል (ለበለጠ መረጃ የ 0 ወይም 1 መመስረቻ ፅሁፍን ይመልከቱ)። በዚህ አክሲዮን ማህበር የቀረበውን መፍትሄ ባጭሩ እንደሚከተለው አስቀምጠነዋል። 
\begin{enumerate}
\item[(1)] ማንበብን ወይም ጥናትን መዝናኛና ገንዘብ ማግኛ እንዲሁም ደግሞ ሽልማት የሚያስገኝ ማድረግ። ከማጥኛ ወይም አዲስ እውቀትን ከማግኛ  ዘዴወች ውስጥ አንደኛው ነገሮችን በተመሳሳያቸው በማዛመድ ማወቅ ነው። ለምሳሌ የአንድ እንግሊዘኛ ቃል ብዙ ተመሳሳይ ቃላቶች አሉት። እነሱን በማዛመድ ለመሸምደድ መሞከር ጥሩ ከሚባሉት ዘዴወች ውስጥ አንዱ ነው። ግን ደግሞ ይሄን ልምምዶሽ አይረሴ ለማድረግ በጨዋታ መልክ ሆኖ በቡድን እየተዝናኑና እየተወያዩ ሲሆን ተመራጭ ያደርገዋል።
ካርድ በማዘጋጀት የእንግሊዘኛ ቃላቶችን ማጥናት በሚል ዙሪያ የተጠኑ ሳይንሳዊ ጥናቶች አሉ (ለምሳሌ፣ እነዚህን ይመልከቱ፣ 
\cite{aslan2011teaching,azabdaftari2012comparing,bryson2012using,kosim2013improving,
 nikoopour2014vocabulary,
nugroho2012improving,
saputri2017improving,senzaki2017reinventing,sitompul2013teaching,
wahyuni2014flashcards})
\item[(2)] ነገሮችን በአይነት አይነታቸው እያዛማዱ ማወቅ ያመራምራል፣ ጠያቂ ያደርጋል፣ ከጓደኛ ጋር ያከራክራል፣ ማመሳከሪያ መፅሃፍ ፍለጋ እስከመሄድ ድረስ ያደርሳል። እናም በዚህ መልክ ሲሆን ያን ነገር ለመርሳት ብዙ ጊዜ ይጨርሳል። 
\item[(3)] ማዛመድን ደግሞ ከጓደኛ ጋር ሆነው እየተዝናኑ በጨዋታ መልክ ካደረጉትና እውቀትንና ማወቅን ለማበረታት ደግሞ ለአሸናፊው ጉርሻ በመስጠት ከሆነ ጨዋታውም ተወዳጅ ይሆናል ማለት ነው።
\item[(4)] ከላይ ከ1-3 የተጠቀሱትን መፍትሔወች ለማከናወን የተለያዩ አይነት አዝናኝ ጨዋታወችን ማዘጋጀት።
\end{enumerate}
በዚህ ወረቀት ውስጥ፣ ለ 0 ወይም 1 ጨዋታ የሚሆን ካርድን አዘጋጅተናል። ያዘጋጀነው ካርድ ለጠቅላላ እውቀት 0.4 የሚሆን ሲሆን ከዚህ በፊት ያልተዘጋጁ ካርዶችን የሚዳስስ ነው። ያዘጋጀነውን የካርዶቹን መረጃ ባጭሩ እንደሚከተለው ገልፀነዋል። የመርፌ ብዛት=6 እና k=4 ቢሆኑ። ስለዚህ n=8*4+6=38 ይሆናል። ስለዚህ አጫዋች ካርዶችን ጨምሮ ባጠቃላይ 38 ካርዶች አሉ። ተጫዋች ካርዶች፤ $38-6=32$ ካርዶች ይሆናሉ፤ 32 ደግሞ የ 8 ብዜት ነው (ለበለጠ መረጃ የዜሮ ወይም አንድ መመስረቻ ፅሁፍን ይመልከቱ)።  አጫዋች ካርዶች የሚከተሉት ናቸው፤ extraneous፣ Provocative፣ Brusque፣ Momentum፣ Fraction Addition Formula እና Equation of a circle ናቸው።
\section{\begin{geez}አጫዋች ካርዶች (Jester Cards)\end{geez}}
\label{S:2}
\begin{definition}[Extraneous]Something that is not essential or relevant to the matter at hand. (see, \cite{dictionary2002merriam}).\\
Example: In solving the equation , the solutions are valid, but if we square both sides of an equation like and get , the solution would be extraneous.
\end{definition}
\begin{definition}[Provocative]Causing a strong reaction, often in a deliberate way; intended to stimulate thought, debate, or controversy.  (see,  \cite{dictionary1989oxford}).\\
Example: A provocative article questioning the validity of established scientific theories may spark debate in the academic community.
\end{definition}
\begin{definition}[Brusque]Abrupt, blunt, or curt in manner or speech, often perceived as rude. (see, \cite{miller2013cambridge}).\\
Example:  When asked about the project's progress, his brusque reply was, ``It's done. Stop asking."
\end{definition}
\begin{definition}[Momentum] In physics, momentum is the product of an object's mass and velocity, given by . More generally, it refers to the force or energy gained by a moving object or idea. (see, \cite{halliday2013fundamentals}).\\
Example: A moving car of mass 1 kg traveling at 3 $m/s$ has a momentum of  $3 kg(m/s)$.
\end{definition}
\begin{definition}[Fraction Addition Formula] The sum of two fractions $\frac{a}{b}$, $\frac{c}{d}$ (where $b,d\ne 0$) is given by:
$$\frac{a}{b}+\frac{c}{d}=\frac{ad+bc}{bd}.$$
Example: $$\frac{1}{2} + \frac{3}{4} = \frac{(1 \times 4) + (3 \times 2)}{2 \times 4} = \frac{4+6}{8} = \frac{10}{8} = \frac{5}{4}.$$
\end{definition}
\begin{definition}[Equation of a circle] The equation $(x-h)^2+(y-k)^2=r$ represents a circle centred at $(h,k)$ with radius $r$ (see, \cite{stewart2012calculus}).\\
Example:
The equation $(x-1)^2+(y-2)^2=1$ represents a circle centred at $(1,2)$ with radius $1$.
\end{definition}
\section{\begin{geez}ተጫዋች ካርዶች ከነአጫዋቻቸው (Player Cards with their Jester)\end{geez}}
\label{S:3}
\begin{enumerate}
\item extraneous=irrelevant=unrelated=unconnected=inapplicable=peripheral=immaterial.
\item provocative=annoying=irritating=goading=vexing=galling=exasperating.
\item brusque=curt=abrupt=blunt=short=sharp=terse=brisk=crisp=clipped=monosyllabic=indelicate=tactless=offhand=snappish=peremptory.
\item momentum=(mass)x(velocity)=the force required to bring the object to a stop in a unit length of time.
\item $\frac{a}{b}+\frac{c}{d}=\frac{ad+bc}{bd}$, $b,d\ne 0$.
\item $(x-1)^2+(y-2)^2=1$=unit circle centered at $(1,2)$=See Figure \ref{figure1} below.
\end{enumerate}
\begin{figure}[h!]
\begin{center}
\includegraphics[width = 3.4cm]{circle}
\end{center}
\caption{Circle centered at the $(1,2)$ with radius 1.}
\label{figure1}
\end{figure}
\printbibliography
\end{document}
