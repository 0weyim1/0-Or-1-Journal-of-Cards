\documentclass[a4paper,12pt]{article}
\usepackage[margin=1in,left=1in,includefoot]{geometry}
\usepackage[margin=1in,left=1in,includefoot]{geometry}
\usepackage[tbtags]{amsmath}%%%%%
%%%%%%%%%%%%%%%%%%%%%%%%%
 \usepackage{amsmath}
 \usepackage{amsfonts}
 \usepackage{amssymb}
\usepackage{mathtools}
\usepackage{amsthm}
\everymath{\displaystyle}
\usepackage[T1]{fontenc}
\usepackage{mathpazo}%%palatino font for text
\usepackage[euler-digits,euler-hat-accent]{eulervm}%euler for math font
%%%%%%%%%%%%%%%%%%%%%%%%%%%%%%%%%%%%
%%%%%%%%%%%%%%%%%%%%%%%%%%%%%%%%%%%%
\usepackage{ragged2e}%%%%%%%%%%%%%%for \justify
%%%%%%%%%%%%%%%%%%%%%%%%%%%%
\usepackage[dvipsnames]{xcolor}
\usepackage{xcolor}
\usepackage{booktabs,tabularx}
\usepackage{multirow}
\usepackage{tikz}
\usepackage{graphicx} % Required for including images
\usepackage[font=small,labelfont=bf]{caption} % Required for specifying captions to tables and figures
%%%%%%%%%%%%%%%%%%%%%%%%%%%%%%%%
\usepackage[colorlinks=true]{hyperref}%
 %%%%%%%%%%%%%%%%%%%%%%%%%%%%%%%%%%%%%
\usepackage{fontspec}
\usepackage[ethiop,main=english]{babel}
\newfontface{\geezfont}{FreeSerif}
\newenvironment{geez}{\geezfont}{}
\lccode`፡=`፡  \catcode`፡=11
\lccode`።=`። \catcode`።=11
\babelprovide[import,
  onchar = fonts ids,
  typography/intraspace = 0 .1 0,
  typography/linebreaking = s, 
  characters/ranges = 1200..139F 2D80..2DDF AB00..AB2F,
  ]{amharic}
\babelfont[amharic]{rm}{FreeSerif}
%%%%%%%%%%%%%%%%%%%%%%%%%%%%%%%%%%%%%%%%%%%%%%%%
%%%%%%%%%%%%%%%%%%%%%%%%%%%%%%%%%%%%%%%%%%%%%%%%%%%%
\newtheoremstyle{mystyle}%                % Name
  {}%                                     % Space above
  {}%                                     % Space below
  {\itshape}%                             % Body font
  {}%                                     % Indent amount
  {\bfseries}%                            % Theorem head font
  {.}%                                    % Punctuation after theorem head
  { }%                                    % Space after theorem head, ' ', or \newline
  {}%                                     % Theorem head spec (can be left empty, meaning `normal')
%%%%%%%%%%%%%%%%%%%%%%%%%%%%%%%%%%%%%%%%%%%%%%%%%%%%%%%%%%%%%%%%%%%%%
\theoremstyle{mystyle}
\newtheorem{theorem}{Theorem}
\newtheorem{proposition}{Proposition}
\newtheorem{lemma}{Lemma}
\newtheorem{corollary}{Corollary}
\newtheorem{example}{Example}
\newtheorem{solution}{Solution}
\newtheorem{conclusion}{Conclusion}
\newtheorem{definition}{Definition}
\newtheorem{remark}{Remark}
\newtheorem{amharicdefinition}{\begin{geez}ትርጉም\end{geez}}
%%%%%%%%%%%%%%%%%%%%%%%%%%%%%%%
%%%%%%%%%%%%%%%%%%%%%%%%%%%%%%%%%%%%%%%%%%%%%%%%%
\numberwithin{equation}{section}
\numberwithin{theorem}{section}
\numberwithin{proposition}{section}
\numberwithin{example}{section}
\numberwithin{remark}{section}
\numberwithin{lemma}{section}
\numberwithin{corollary}{section}
\numberwithin{definition}{section}
\numberwithin{amharicdefinition}{section}
%%%%%%%%%%%%%%%%%%%%%%%%%%%%%%%%%%%%%%%%%%%%%%%%%%
%%%%%%%%%%%%%%%%%%%%%%%%%%
\usepackage[shortlabels]{enumitem}
\usepackage{soul}%%for highlighting texts and equations begin inside $$..\hl{some text here}
\newcommand{\mathcolorbox}[2]{\colorbox{#1}{$\displaystyle #2$}}%%to highlight math equations which is inside \begin{equation}...\end{equation}
\usepackage{authblk}
\title{
{\large General Knowledge 0.7 For Pin Number 6}
}
\author[1,2,$*$]{\small Dagnachew Jenber}
\author[3]{Fikreab Habtamu}
%\author[3]{third author}
%\author[4]{fourth author}
\affil[1]{ Department of Mathematics, Bahir Dar University, Bahir Dar, Ethiopia.}
\affil[2]{Department of Mathematics, Addis Ababa Science and Technology University, Addis Ababa, Ethiopia.}
\affil[3]{Department of Chemistry, Bahir Dar University, Bahir Dar, Ethiopia.}
%\affil[4]{fourth affilation}
\affil[$*$]{Corresponding author: Dagnachew Jenber, dagnachew.Jenber@aastu.edu.et}
\setcounter{Maxaffil}{0}
\renewcommand\Affilfont{\itshape\small}
\usepackage{xhfill}
\setlength{\parindent}{0pt}
%%%%%%%%%%%%%%%%%%%%%%%%%%%%%%%%%%%%%%%%%%%%%
%%%%%%%%%%%%%%%%%%%%%%%%%%%%%%%%%%%%%%%%%%%%%%%%%%%%%%%%%%
\usepackage[backend=bibtex,maxnames=1000,minnames=10,maxalphanames=1000,
minalphanames=10,style=numeric,sorting=anyt,firstinits=true]{biblatex}
\DeclareNameAlias{default}{last-first}
\addbibresource{volume-1-article-25.1.5.bib}
\renewbibmacro{in:}{}
%%%%%%%%%%%%%%%%%%%%%%%%%%%%%
\usepackage[tbtags]{amsmath}%%%%%https://tex.stackexchange.com/questions/368353/using-equation-split-how-can-i-ensure-that-only-the-last-equation-is-numbered%%%%
%%%%%%%%%%%%%%%%%%%%%%%%%
\usepackage{geometry}
\usepackage{graphicx}
\makeatletter         
\def\@maketitle{
\raggedright
\begin{center}
{\large \bfseries \sffamily \@title }\\[1.5ex]
{  \@author}\\[8ex]
\end{center}}
\makeatother
%%%%%%%%%%%%%%%%%%%%%%%%%%%%%%%%%%%%%%%%%%%%%%%
\makeatletter
\renewcommand\tableofcontents{%
  \null\hfill\textbf{\Large\contentsname}\hfill\null\par
  \@mkboth{\MakeUppercase\contentsname}{\MakeUppercase\contentsname}%
  \@starttoc{toc}%
}
\makeatother
%%%%%%%%%%%%%%%%%%%%%%%%%%%%%%%%%%%%%%%%%%%%%%%%%%%
%%%%%%%%%%%%%%%%%%%%%%%%%%%%%%%%%%%%%%%%%%%%%%%%%%%%%
\usepackage{hyperref}% http://ctan.org/pkg/hyperref
%%%%%%%%%%%%%%%%%%%%%%%%%%%%%%%%%%%%

%%%%%%%%%%%%%%%%%%%%%%%%%%%%%%%%%%%%%%
%%%%%%%%%%%%%%%%%%%%%%%%%%%%%%%%%%%%%%%%
\begin{document}
\maketitle
\fontfamily{kpfonts}
\hypersetup{
  colorlinks,
  citecolor=red,
  linkcolor=red,
  urlcolor=blue}

  \hypersetup{
  citebordercolor=red,
  filebordercolor=red,
  linkbordercolor=blue
}
\centering
{\bf Abstract}
\justify
This work presents 62 number of cards from different discplines focused on english, physics and mathematics subject. The jester cards are prudent, lascivious, fatuous, force, cosine of $n\pi$, $n\in \mathbb{N}$ and exponent rule.
\renewcommand{\thefootnote}{}%remove numbering for the footnote
\footnotetext{\scriptsize \textcopyright\ The Author(s) 2025. Open Access This article is licensed under a Creative Commons Attribution-NonCommercial-NoDerivatives
4.0 International License, which permits any non-commercial use, sharing, distribution and reproduction in any medium or format, as long as you give appropriate credit to the original author(s) and the source, provide a link to the Creative Commons licence, and indicate if you modified the licensed material. You do not have permission under this licence to share adapted material derived
from this article or parts of it. The images or other third party material in this article are included in the article’s Creative Commons licence, unless indicated otherwise in a credit line to the material. If material is not included in the article’s Creative Commons licence and your intended use is not permitted by statutory regulation or exceeds the permitted use, you will need to obtain
permission directly from the copyright holder. To view a copy of this licence, visit \url{https://creativecommons.org/licenses/by-nc-nd/4.0/}.}
\renewcommand{\thefootnote}{\arabic{footnote}}%%restore default numbering for future footnotes
\section{\begin{geez}መግቢያ\end{geez}}
\label{S:2}
አሁን ባለንበት ዘመን የአንባቢያን ማህበረሰብ እየቀነሰ መምጣት አሳሳቢ ደረጃ ላይ ደርሷል። በብዙ ምክኒያት ሰወች ቁጭ ብለው
ማንበብ የተውበት ጊዜ ነው። ለምሳሌ ጠቃሚ ያልሆነ ሶሻል
ሚዲያ ላይና በአልባሌ ቦታወች ጊዜን ማጥፋት ከብዙወቹ ትንሾቹ ምክኒያቶች ናቸው። በ2017 ዓ.ም ዳኛቸው ለዚህ የሚሆን መፍትሄ ብሎ ያቀረበው 0 ወይም 1 ጨዋታ በሚል ርእስ የተዘጋጀ ትልቅ አክሲዮን ማህበር አለ። ይህ አክሲዮን ማህበር ከላይ የተጠቀሰውን ችግር በሚከተሉት መልኩ መፍታት ይቻላል ብሎ ያምናል። በዚህ ፅሁፍ ውስጥ የተካተተው መፍትሄ አሳማኝ ሆኖ አግኝተነዋል (ለበለጠ መረጃ የ 0 ወይም 1 መመስረቻ ፅሁፍን ይመልከቱ)። በዚህ አክሲዮን ማህበር የቀረበውን መፍትሄ ባጭሩ እንደሚከተለው አስቀምጠነዋል። 
\begin{enumerate}
\item[(1)] ማንበብን ወይም ጥናትን መዝናኛና ገንዘብ ማግኛ እንዲሁም ደግሞ ሽልማት የሚያስገኝ ማድረግ። ከማጥኛ ወይም አዲስ እውቀትን ከማግኛ  ዘዴወች ውስጥ አንደኛው ነገሮችን በተመሳሳያቸው በማዛመድ ማወቅ ነው። ለምሳሌ የአንድ እንግሊዘኛ ቃል ብዙ ተመሳሳይ ቃላቶች አሉት። እነሱን በማዛመድ ለመሸምደድ መሞከር ጥሩ ከሚባሉት ዘዴወች ውስጥ አንዱ ነው። ግን ደግሞ ይሄን ልምምዶሽ አይረሴ ለማድረግ በጨዋታ መልክ ሆኖ በቡድን እየተዝናኑና እየተወያዩ ሲሆን ተመራጭ ያደርገዋል።
ካርድ በማዘጋጀት የእንግሊዘኛ ቃላቶችን ማጥናት በሚል ዙሪያ የተጠኑ ሳይንሳዊ ጥናቶች አሉ (ለምሳሌ፣ እነዚህን ይመልከቱ፣ 
\cite{aslan2011teaching,azabdaftari2012comparing,bryson2012using,kosim2013improving,
 nikoopour2014vocabulary,
nugroho2012improving,
saputri2017improving,senzaki2017reinventing,sitompul2013teaching,
wahyuni2014flashcards})
\item[(2)] ነገሮችን በአይነት አይነታቸው እያዛማዱ ማወቅ ያመራምራል፣ ጠያቂ ያደርጋል፣ ከጓደኛ ጋር ያከራክራል፣ ማመሳከሪያ መፅሃፍ ፍለጋ እስከመሄድ ድረስ ያደርሳል። እናም በዚህ መልክ ሲሆን ያን ነገር ለመርሳት ብዙ ጊዜ ይጨርሳል። 
\item[(3)] ማዛመድን ደግሞ ከጓደኛ ጋር ሆነው እየተዝናኑ በጨዋታ መልክ ካደረጉትና እውቀትንና ማወቅን ለማበረታት ደግሞ ለአሸናፊው ጉርሻ በመስጠት ከሆነ ጨዋታውም ተወዳጅ ይሆናል ማለት ነው።
\item[(4)] ከላይ ከ1-3 የተጠቀሱትን መፍትሔወች ለማከናወን የተለያዩ አይነት አዝናኝ ጨዋታወችን ማዘጋጀት።
\end{enumerate}
በዚህ ወረቀት ውስጥ፣ ለ 0 ወይም 1 ጨዋታ የሚሆን ካርድን አዘጋጅተናል። ያዘጋጀነው ካርድ ለጠቅላላ እውቀት 0.7 የሚሆን ሲሆን ከዚህ በፊት ያልተዘጋጁ ካርዶችን የሚዳስስ ነው። ያዘጋጀነውን የካርዶቹን መረጃ ባጭሩ እንደሚከተለው ገልፀነዋል። የመርፌ ብዛት=6 እና k=7 ቢሆኑ። ስለዚህ n=8*7+6=62 ይሆናል። ስለዚህ አጫዋች ካርዶችን ጨምሮ ባጠቃላይ 62 ካርዶች አሉ። ተጫዋች ካርዶች፤ $62-6=56$ ካርዶች ይሆናሉ፤ 56 ደግሞ የ 8 ብዜት ነው (ለበለጠ መረጃ የዜሮ ወይም አንድ መመስረቻ ፅሁፍን ይመልከቱ)።  አጫዋች ካርዶች የሚከተሉት ናቸው፤ prudent፣ lascivious፣ fatuous፣ force፣ cosine of $n\pi$፣ እና exponent rule ናቸው።
\section{\begin{geez}አጫዋች ካርዶች (Jester Cards)\end{geez}}
\label{S:2}
\begin{definition}[Prudent]Acting with or showing care and thought for the future. (see, \cite{dictionary2002merriam}).\\
Example: A prudent investor carefully studies market trends before making any financial decisions.
\end{definition}
\begin{definition}[Lascivious]Feeling or revealing an overt and often offensive sexual desire.  (see,  \cite{dictionary1989oxford}).\\
Example: The court ruled against the defendant due to his lascivious behavior toward his colleagues.
\end{definition}
\begin{definition}[Fatuous]Silly and pointless; lacking intelligence or thought. (see, \cite{dictionary2002merriam}).\\
Example: His fatuous remarks during the meeting annoyed his colleagues, as they added nothing meaningful to the discussion.
\end{definition}
\begin{definition}[Force (Physics)]A vector quantity that causes an object to undergo a change in motion (acceleration), given by Newton's Second Law: $F=ma$, where $F$ is force, $m$ is mass, and $a$ is acceleration.  (see, \cite{serway2018current}).\\
Example: The gravitational force acting on a falling apple pulls it toward the ground.
\end{definition}
\begin{definition}[Cosine Function (Mathematics)] A trigonometric function defined as the ratio of the adjacent side to the hypotenuse in a right-angled triangle. For more see \cite{stewart2012calculus}.\\
Example: The cosine of 60 degrees is 0.5.
\end{definition}
\begin{definition}[Exponent Rule (Mathematics)] (see, \cite{wilczynski1916college}).The rules governing exponentiation, such as:
\begin{itemize}
\item $a^m.a^n=a^{m+n}$ (Product Rule)
\item $\frac{a^m}{a^n}=a^{m-n}$ (Quotient Rule)
\item $(a^m)^n=a^{mn}$(Power Rule)
\end{itemize}
Example: Using the power rule, $(2^3)^4=2^{3(4)}=2^{12}$.
\end{definition}
\section{\begin{geez}ተጫዋች ካርዶች ከነአጫዋቻቸው (Player Cards with their Jester)\end{geez}}
\label{S:3}
\begin{enumerate}
\item prudent=careful=reasonable=cautious=circumspect=judicious=vigilant.
\item lascivious=lecherous=lewd=lustful=wanton=indecent=ribald=unchaste=obscene=salacious=depraved=libidinous=licentious=lubricious=prurient.
\item fatuous=silly=foolish=inane=witless=empty headed=pointless=preposterous=ludicrous=imbecilic=asinine=vacuous=frivolous=trivial=shallow=senseless=useless=purposeless=flippant=waggish=superficial=facetious=superficial=whimsical=skittish=flighty=flippant=vacuous.
\item force=(mass)x(acceleration)=a push or pull on an object=has the SI unit newton(N).
\item cosine of $n\pi$, $n\in \mathbb{N}$=$\cos(n\pi)$, $n\in \mathbb{N}$=$(-1)^n$, $n\in \mathbb{N}$=example of alternating sequence.
\item $a^{x+y}(bc)^{z}d^{-n}\left(\frac{e}{f}\right)^m(g^m)^n$=$(a^{x})(a^{y})(bc)^{z}d^{-n}\left(\frac{e}{f}\right)^m(g^{mn})$\\=$(a^{x})(a^{y})(b^{z}c^{z})\left(\frac{1}{d^n}\right)\left(\frac{e}{f}\right)^m(g^{mn})$=$(a^{x})(a^{y})(b^z)(c^z)\left(\frac{1}{d^n}\right)\left(\frac{e^m}{f^m}\right)(g^{mn})$.
\end{enumerate}
\printbibliography
\end{document}
