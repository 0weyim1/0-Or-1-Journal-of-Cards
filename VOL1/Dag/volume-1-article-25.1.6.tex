\documentclass[a4paper,12pt]{article}
\usepackage[margin=1in,left=1in,includefoot]{geometry}
\usepackage[margin=1in,left=1in,includefoot]{geometry}
\usepackage[tbtags]{amsmath}%%%%%
%%%%%%%%%%%%%%%%%%%%%%%%%
 \usepackage{amsmath}
 \usepackage{amsfonts}
 \usepackage{amssymb}
\usepackage{mathtools}
\usepackage{amsthm}
\everymath{\displaystyle}
\usepackage[T1]{fontenc}
\usepackage{mathpazo}%%palatino font for english text
\usepackage[euler-digits,euler-hat-accent]{eulervm}%euler for math font
%%%%%%%%%%%%%%%%%%%%%%%%%%%%%%%%%%%%
%%%%%%%%%%%%%%%%%%%%%%%%%%%%%%%%%%%%
\usepackage{ragged2e}%%%%%%%%%%%%%%for \justify
%%%%%%%%%%%%%%%%%%%%%%%%%%%%
\usepackage[dvipsnames]{xcolor}
\usepackage{xcolor}
\usepackage{booktabs,tabularx}
\usepackage{multirow}
\usepackage{tikz}
\usepackage{graphicx} % Required for including images
\usepackage[font=small,labelfont=bf]{caption} % Required for specifying captions to tables and figures
%%%%%%%%%%%%%%%%%%%%%%%%%%%%%%%%
\usepackage[colorlinks=true]{hyperref}%
 %%%%%%%%%%%%%%%%%%%%%%%%%%%%%%%%%%%%%
\usepackage{fontspec}
\usepackage[ethiop,main=english]{babel}
\newfontface{\geezfont}{FreeSerif}
\newenvironment{geez}{\geezfont}{}
\lccode`፡=`፡  \catcode`፡=11
\lccode`።=`። \catcode`።=11
\babelprovide[import,
  onchar = fonts ids,
  typography/intraspace = 0 .1 0,
  typography/linebreaking = s, 
  characters/ranges = 1200..139F 2D80..2DDF AB00..AB2F,
  ]{amharic}
\babelfont[amharic]{rm}{FreeSerif}
%%%%%%%%%%%%%%%%%%%%%%%%%%%%%%%%%%%%%%%%%%%%%%%%
%%%%%%%%%%%%%%%%%%%%%%%%%%%%%%%%%%%%%%%%%%%%%%%%%%%%
\newtheoremstyle{mystyle}%                % Name
  {}%                                     % Space above
  {}%                                     % Space below
  {\itshape}%                             % Body font
  {}%                                     % Indent amount
  {\bfseries}%                            % Theorem head font
  {.}%                                    % Punctuation after theorem head
  { }%                                    % Space after theorem head, ' ', or \newline
  {}%                                     % Theorem head spec (can be left empty, meaning `normal')
%%%%%%%%%%%%%%%%%%%%%%%%%%%%%%%%%%%%%%%%%%%%%%%%%%%%%%%%%%%%%%%%%%%%%
\theoremstyle{mystyle}
\newtheorem{theorem}{Theorem}
\newtheorem{proposition}{Proposition}
\newtheorem{lemma}{Lemma}
\newtheorem{corollary}{Corollary}
\newtheorem{example}{Example}
\newtheorem{solution}{Solution}
\newtheorem{conclusion}{Conclusion}
\newtheorem{definition}{Definition}
\newtheorem{remark}{Remark}
\newtheorem{amharicdefinition}{\begin{geez}ትርጉም\end{geez}}
%%%%%%%%%%%%%%%%%%%%%%%%%%%%%%%
%%%%%%%%%%%%%%%%%%%%%%%%%%%%%%%%%%%%%%%%%%%%%%%%%
\numberwithin{equation}{section}
\numberwithin{theorem}{section}
\numberwithin{proposition}{section}
\numberwithin{example}{section}
\numberwithin{remark}{section}
\numberwithin{lemma}{section}
\numberwithin{corollary}{section}
\numberwithin{definition}{section}
\numberwithin{amharicdefinition}{section}
%%%%%%%%%%%%%%%%%%%%%%%%%%%%%%%%%%%%%%%%%%%%%%%%%%
%%%%%%%%%%%%%%%%%%%%%%%%%%
\usepackage[shortlabels]{enumitem}
\usepackage{soul}%%for highlighting texts and equations begin inside $$..\hl{some text here}
\newcommand{\mathcolorbox}[2]{\colorbox{#1}{$\displaystyle #2$}}%%to highlight math equations which is inside \begin{equation}...\end{equation}
\usepackage{authblk}
\title{
{\large General Knowledge 0.3 For Pin Number 6}
}
\author[1,2,$*$]{\small Dagnachew Jenber}
\author[3]{Aman Matebie}
%\author[4]{fourth author}
\affil[1]{ Department of Mathematics, Bahir Dar University, Bahir Dar, Ethiopia.}
\affil[2]{Department of Mathematics, Addis Ababa Science and Technology University, Addis Ababa, Ethiopia.}
\affil[3]{Department of English, Debremarkos University, Debremarkos, Ethiopia.}
%\affil[4]{fourth affilation}
\affil[$*$]{Corresponding author: Dagnachew Jenber, dagnachew.jenber@aastu.edu.et}
\setcounter{Maxaffil}{0}
\renewcommand\Affilfont{\itshape\small}
\usepackage{xhfill}
\setlength{\parindent}{0pt}
%%%%%%%%%%%%%%%%%%%%%%%%%%%%%%%%%%%%%%%%%%%%%
%%%%%%%%%%%%%%%%%%%%%%%%%%%%%%%%%%%%%%%%%%%%%%%%%%%%%%%%%%
\usepackage[backend=bibtex,maxnames=1000,minnames=10,maxalphanames=1000,
minalphanames=10,style=numeric,sorting=anyt,firstinits=true]{biblatex}
\DeclareNameAlias{default}{last-first}
\addbibresource{volume-1-article-25.1.6.bib}
\renewbibmacro{in:}{}
%%%%%%%%%%%%%%%%%%%%%%%%%%%%%
\usepackage[tbtags]{amsmath}%%%%%https://tex.stackexchange.com/questions/368353/using-equation-split-how-can-i-ensure-that-only-the-last-equation-is-numbered%%%%
%%%%%%%%%%%%%%%%%%%%%%%%%
\usepackage{geometry}
\usepackage{graphicx}
\makeatletter         
\def\@maketitle{
\raggedright
\begin{center}
{\large \bfseries \sffamily \@title }\\[1.5ex]
{  \@author}\\[8ex]
\end{center}}
\makeatother
%%%%%%%%%%%%%%%%%%%%%%%%%%%%%%%%%%%%%%%%%%%%%%%
\makeatletter
\renewcommand\tableofcontents{%
  \null\hfill\textbf{\Large\contentsname}\hfill\null\par
  \@mkboth{\MakeUppercase\contentsname}{\MakeUppercase\contentsname}%
  \@starttoc{toc}%
}
\makeatother
%%%%%%%%%%%%%%%%%%%%%%%%%%%%%%%%%%%%%%%%%%%%%%%%%%%
%%%%%%%%%%%%%%%%%%%%%%%%%%%%%%%%%%%%%%%%%%%%%%%%%%%%%
\usepackage{hyperref}% http://ctan.org/pkg/hyperref
%%%%%%%%%%%%%%%%%%%%%%%%%%%%%%%%%%%%

%%%%%%%%%%%%%%%%%%%%%%%%%%%%%%%%%%%%%%
%%%%%%%%%%%%%%%%%%%%%%%%%%%%%%%%%%%%%%%%
\begin{document}
\maketitle
\fontfamily{kpfonts}
\hypersetup{
  colorlinks,
  citecolor=red,
  linkcolor=red,
  urlcolor=blue}

  \hypersetup{
  citebordercolor=red,
  filebordercolor=red,
  linkbordercolor=blue
}
\centering
{\bf Abstract}
\justify
This work presents 30 number of cards from different discplines focused on english, physics and mathematics subject. The jester cards are Pervasive, Pristine, Deleterious, Impulse, $\mathbb{R}$ and
$P(x)=a\textsubscript{n}x^n+a\textsubscript{n-1} x^{n-1}+\cdots+a\textsubscript {1} x+a\textsubscript {0},$
where $n\in \mathbb{N}$, $a\textsubscript{n}\ne 0$.
\renewcommand{\thefootnote}{}%remove numbering for the footnote
\footnotetext{\scriptsize \textcopyright\ The Author(s) 2025. Open Access This article is licensed under a Creative Commons Attribution-NonCommercial-NoDerivatives
4.0 International License, which permits any non-commercial use, sharing, distribution and reproduction in any medium or format, as long as you give appropriate credit to the original author(s) and the source, provide a link to the Creative Commons licence, and indicate if you modified the licensed material. You do not have permission under this licence to share adapted material derived
from this article or parts of it. The images or other third party material in this article are included in the article’s Creative Commons licence, unless indicated otherwise in a credit line to the material. If material is not included in the article’s Creative Commons licence and your intended use is not permitted by statutory regulation or exceeds the permitted use, you will need to obtain
permission directly from the copyright holder. To view a copy of this licence, visit \url{https://creativecommons.org/licenses/by-nc-nd/4.0/}.}
\renewcommand{\thefootnote}{\arabic{footnote}}%%restore default numbering for future footnotes
\section{\begin{geez}መግቢያ\end{geez}}
\label{S:2}
አሁን ባለንበት ዘመን የአንባቢያን ማህበረሰብ እየቀነሰ መምጣት አሳሳቢ ደረጃ ላይ ደርሷል። በብዙ ምክኒያት ሰወች ቁጭ ብለው
ማንበብ የተውበት ጊዜ ነው። ለምሳሌ ጠቃሚ ያልሆነ ሶሻል
ሚዲያ ላይና በአልባሌ ቦታወች ጊዜን ማጥፋት ከብዙወቹ ትንሾቹ ምክኒያቶች ናቸው። በ2017 ዓ.ም ዳኛቸው ለዚህ የሚሆን መፍትሄ ብሎ ያቀረበው 0 ወይም 1 ጨዋታ በሚል ርእስ የተዘጋጀ ትልቅ አክሲዮን ማህበር አለ። ይህ አክሲዮን ማህበር ከላይ የተጠቀሰውን ችግር በሚከተሉት መልኩ መፍታት ይቻላል ብሎ ያምናል። በዚህ ፅሁፍ ውስጥ የተካተተው መፍትሄ አሳማኝ ሆኖ አግኝተነዋል (ለበለጠ መረጃ የ 0 ወይም 1 መመስረቻ ፅሁፍን ይመልከቱ)። በዚህ አክሲዮን ማህበር የቀረበውን መፍትሄ ባጭሩ እንደሚከተለው አስቀምጠነዋል። 
\begin{enumerate}
\item[(1)] ማንበብን ወይም ጥናትን መዝናኛና ገንዘብ ማግኛ እንዲሁም ደግሞ ሽልማት የሚያስገኝ ማድረግ። ከማጥኛ ወይም አዲስ እውቀትን ከማግኛ  ዘዴወች ውስጥ አንደኛው ነገሮችን በተመሳሳያቸው በማዛመድ ማወቅ ነው። ለምሳሌ የአንድ እንግሊዘኛ ቃል ብዙ ተመሳሳይ ቃላቶች አሉት። እነሱን በማዛመድ ለመሸምደድ መሞከር ጥሩ ከሚባሉት ዘዴወች ውስጥ አንዱ ነው። ግን ደግሞ ይሄን ልምምዶሽ አይረሴ ለማድረግ በጨዋታ መልክ ሆኖ በቡድን እየተዝናኑና እየተወያዩ ሲሆን ተመራጭ ያደርገዋል።
ካርድ በማዘጋጀት የእንግሊዘኛ ቃላቶችን ማጥናት በሚል ዙሪያ የተጠኑ ሳይንሳዊ ጥናቶች አሉ (ለምሳሌ፣ እነዚህን ይመልከቱ፣ 
\cite{aslan2011teaching,azabdaftari2012comparing,bryson2012using,kosim2013improving,
 nikoopour2014vocabulary,
nugroho2012improving,
saputri2017improving,senzaki2017reinventing,sitompul2013teaching,
wahyuni2014flashcards})
\item[(2)] ነገሮችን በአይነት አይነታቸው እያዛማዱ ማወቅ ያመራምራል፣ ጠያቂ ያደርጋል፣ ከጓደኛ ጋር ያከራክራል፣ ማመሳከሪያ መፅሃፍ ፍለጋ እስከመሄድ ድረስ ያደርሳል። እናም በዚህ መልክ ሲሆን ያን ነገር ለመርሳት ብዙ ጊዜ ይጨርሳል። 
\item[(3)] ማዛመድን ደግሞ ከጓደኛ ጋር ሆነው እየተዝናኑ በጨዋታ መልክ ካደረጉትና እውቀትንና ማወቅን ለማበረታት ደግሞ ለአሸናፊው ጉርሻ በመስጠት ከሆነ ጨዋታውም ተወዳጅ ይሆናል ማለት ነው።
\item[(4)] ከላይ ከ1-3 የተጠቀሱትን መፍትሔወች ለማከናወን የተለያዩ አይነት አዝናኝ ጨዋታወችን ማዘጋጀት።
\end{enumerate}
በዚህ ወረቀት ውስጥ፣ ለ 0 ወይም 1 ጨዋታ የሚሆን ካርድን አዘጋጅተናል። ያዘጋጀነው ካርድ ለጠቅላላ እውቀት 0.3 የሚሆን ሲሆን ከዚህ በፊት ያልተዘጋጁ ካርዶችን የሚዳስስ ነው። ያዘጋጀነውን የካርዶቹን መረጃ ባጭሩ እንደሚከተለው ገልፀነዋል። የመርፌ ብዛት=6 እና k=3 ቢሆኑ። ስለዚህ n=8*3+6=30 ይሆናል። ስለዚህ አጫዋች ካርዶችን ጨምሮ ባጠቃላይ 30 ካርዶች አሉ። ተጫዋች ካርዶች፤ $30-6=24$ ካርዶች ይሆናሉ፤ 24 ደግሞ የ 8 ብዜት ነው (ለበለጠ መረጃ የዜሮ ወይም አንድ መመስረቻ ፅሁፍን ይመልከቱ)።  አጫዋች ካርዶች የሚከተሉት ናቸው፤ Pervasive፣ Pristine፣ Deleterious፣ Impulse፣  $\mathbb{R}$ እና  $$P(x)=a\textsubscript {n}x^n+a\textsubscript {n-1}x^{n-1}+\cdots+a\textsubscript {1}x+a\textsubscript {0},$$ where $n\in \mathbb{N}$, $a\textsubscript {n}\ne 0$.
\section{\begin{geez}አጫዋች ካርዶች (Jester Cards)\end{geez}}
\label{S:2}
\begin{definition}[Pervasive] Something that is widespread or present throughout a particular area, group, or system.. (see, \cite{dictionary2002merriam}).\\
Example: The influence of social media is pervasive in modern society, affecting communication, business, and politics.
\end{definition}
\begin{definition}[Pristine] In its original condition; unspoiled; pure. It can also mean clean and fresh as if new.  (see,  \cite{dictionary1989oxford}).\\
Example: The hikers were amazed by the pristine beauty of the untouched forest.
\end{definition}
\begin{definition}[Deleterious] Causing harm or damage. (see, \cite{dictionary2002merriam}).\\
Example: Prolonged exposure to radiation has deleterious effects on human health.
\end{definition}
\begin{definition}[Impulse] A sudden strong urge or desire to act; in physics, the product of force and the time over which it acts, changing an object's momentum. (see, \cite{serway2018current}).\\
Example: Psychological: ``She resisted the impulse to buy unnecessary items."
Physics: ``The impulse applied to a moving object is equal to the change in its momentum."
\end{definition}
\begin{definition}[The set of real numbers] The set of all rational and irrational numbers, including positive numbers, negative numbers, and zero, which can be represented on a number line. The set of real numbers is denoted by $\mathbb{R}$. For more see \cite{rudin2021principles}.\\
Example: Rational numbers: $\frac{3}{4}$, $-2$(which can be written as $\frac{-2}{1}$ ), and 0.75 (which is  $\frac{3}{4}$) are real numbers and irrational numbers: $\sqrt{2},\sqrt{3}$, etc are real numbers.
\end{definition}
\begin{definition}[Polynomial Function] A function of the form
$P(x)=a\textsubscript {n}x^n+a\textsubscript {n-1}x^{n-1}+\cdots+a\textsubscript {1}x+a\textsubscript {0},$ where $n\in \mathbb{N}$, $a\textsubscript {n}\ne 0$ and $a\textsubscript {n},a\textsubscript {n-1},\cdots, a\textsubscript {0}$ are real or complex coefficients.  (see, \cite{lang1987linear}).\\
Example: $f(x)=x^2+x+1$.
\end{definition}
\section{\begin{geez}ተጫዋች ካርዶች ከነአጫዋቻቸው (Player Cards with their Jester)\end{geez}}
\label{S:3}
\begin{enumerate}
\item pervasive=common=universal=extensive=prevalent=penetrating=permeating=ubiquitous.
\item pristine=flawless=clean=pure=fresh=virgin=immaculate.
\item deleterious=harmful=damaging=injurious=detrimental=inimical
\item impulse=(force)x(change in time)=the change in momentum of an object.
\item $\mathbb{R}$=denotes the set of real numbers=is the union of rational and irrational numbers.
\item $P(x)=a\textsubscript {n}x^n+a\textsubscript {n-1}x^{n-1}+\cdots+a\textsubscript {1}x+a\textsubscript {0},$
where $n\in \mathbb{N}$, $a\textsubscript {n}\ne 0$ = a polynomial function with variable $x$=has degree ``n".
\end{enumerate}
\printbibliography
\end{document}
